\documentclass[a4paper,12pt]{report}

\usepackage[margin=2cm, bindingoffset=0cm]{geometry}
\usepackage[english]{babel}
\usepackage[utf8]{inputenc}
\usepackage{fancyhdr}
\usepackage[Bjornstrup]{fncychap}
\usepackage{scalefnt}
\usepackage{mdframed}
\usepackage{xcolor}
\usepackage{graphicx}
\usepackage{capt-of}
\usepackage{indentfirst}
\usepackage{float}
\usepackage{tikz} 
\usepackage{pdfpages}
\usepackage{ragged2e}
\usepackage{titlesec}
\usepackage{listings}
\usepackage{comment}
\usepackage{hyperref}
\hypersetup{colorlinks=true, linkcolor=blue}
\usepackage{array} % Required for new column types
\usepackage{amsmath}
\usepackage{longtable}

\definecolor{darkgreen}{rgb}{0, 0.5, 0}

\lstset{ 
  language=C,
  basicstyle=\ttfamily\footnotesize,
  keywordstyle=\color{blue},
  commentstyle=\color{darkgreen},
  stringstyle=\color{red},
  numbers=left,
  numberstyle=\tiny\color{gray},
  stepnumber=1,
  numbersep=5pt,
  backgroundcolor=\color{lightgray!20},
  showspaces=false,
  showstringspaces=false,
  frame=single,
  breaklines=true,
  breakatwhitespace=true,
  tabsize=4,
  captionpos=b % Caption position set to bottom
}

\renewcommand\arraystretch{1.5} 

%%%% ******************************************************* added
\newcommand{\colortitlechap}{\color[RGB]{0,91,150}} % color of the chapter title <<<
\newcommand{\colornumberchap}{\color[RGB]{1,31,75}} % color of the chapter number <<<<
\newcommand{\colorbackchap}{\colorbox[RGB]{179,205,224}} % color of the background rules <<<

\makeatletter

\renewcommand{\DOCH}{%
    \settowidth{\py}{\CNoV\thechapter}
    \addtolength{\py}{-10pt}% 
    \fboxsep=0pt%
    \colorbackchap{\rule{0pt}{40pt}\parbox[b]{\textwidth}{\hfill}}%
    \kern-\py\raise20pt%
    \hbox{\colornumberchap\CNoV\thechapter}\\%
}

\renewcommand{\DOTI}[1]{%
    \nointerlineskip\raggedright%
    \fboxsep=\myhi%
    \vskip-1ex%
    \colorbackchap{\parbox[t]{\mylen}{\CTV\FmTi{\colortitlechap#1}}}\par\nobreak%
    \vskip 40\p@%
}

\renewcommand{\DOTIS}[1]{%
    \fboxsep=0pt%
    \colorbackchap{\rule{0pt}{40pt}\parbox[b]{\textwidth}{\hfill}}\\%
    \nointerlineskip\raggedright%
    \fboxsep=\myhi%
    \colorbackchap{\parbox[t]{\mylen}{\CTV\FmTi{\colortitlechap#1}}}\par\nobreak%
    \vskip 40\p@%
}

\makeatother
%%*******************************************************************************

\pagestyle{fancy}
\lhead{\scalefont{0.8}\leftmark}
\rhead{\scalefont{0.8}\rightmark}

\definecolor{sfondo1}{RGB}{179,205,224}
\definecolor{sfondo2}{RGB}{228,237,242}

\newcommand{\mybox}[4]{
    \begin{figure}[h]
        \centering
    \begin{tikzpicture}
        \node[anchor=text,text width=\columnwidth-1.2cm, draw, rounded corners, line width=1pt, fill=#3, inner sep=5mm] (big) {\\#4};
        \node[draw, rounded corners, line width=.5pt, fill=#2, anchor=west, xshift=5mm] (small) at (big.north west) {#1};
    \end{tikzpicture}
    \end{figure}
}

% Definisci un colore verde scuro personalizzato
\definecolor{darkgreen}{rgb}{0.0, 0.5, 0.0} % Valori RGB (Red, Green, Blue) da 0.0 a 1.0

% Crea il nuovo comando
\newcommand{\req}[1]{\textcolor{darkgreen}{\texttt{#1}}}

